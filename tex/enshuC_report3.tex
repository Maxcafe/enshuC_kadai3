\documentclass[a4j]{jarticle}

%% "%:"から始まる行を書くと「タグ行」として扱われる。Command+2でタグ行を挿入

\usepackage{mediabb} % pdfファイル挿入用
\usepackage{url}
\usepackage{graphicx}
\usepackage{ascmac}
\usepackage{here}
\usepackage{itembkbx}
\usepackage{subfigure} % for subfigure

\newenvironment{ite}{\begin{itemize}}{\end{itemize}} %箇条書き
\newenvironment{tab}{\begin{table}}{\end{table}} %表
\newenvironment{cen}{\begin{center}}{\end{center}} %中央揃え
\newenvironment{tabu}{\begin{tabular}}{\end{tabular}} %表
\newenvironment{bbox}{\begin{breakbox}}{\end{breakbox}} %四角囲い付け
\newenvironment{scr}{\begin{breakscreen}}{\end{breakscreen}} %丸囲い付け
\newenvironment{bit}{\begin{breakitembox}}{\end{breakitembox}} %囲い付き箇条書き, [c]等必要, {}に題名
\newenvironment{fig}{\begin{figure}}{\end{figure}} %図
\newenvironment{enu}{\begin{enumerate}}{\end{enumerate}} %箇条書き(数字)
\newenvironment{des}{\begin{description}}{\end{description}} %箇条書き(文字)
\newenvironment{bib}{\begin{thebibliography}}{\end{thebibliography}} %参考文献
\newcommand{\cirnum}[1]{\raise0.2ex\hbox{\textcircled{\scriptsize{#1}}}} %丸数字
\newcommand{\ma}{main関数}
\newcommand{\mo}{move関数}
\newcommand{\pa}{path\_print関数}
\newcommand{\pai}{paint関数}

% 講義・演習名
\newcommand{\lectureName}{情報科学演習C}
% レポートの回数
\newcommand{\reportNumber}{3}
% レポートのタイトル
\newcommand{\reportTitle}{マルチプロセス処理}
% 教員名
\newcommand{\teacherName}{小島 英春,内山 彰 教員}
% 課題締め切り日
\newcommand{\deadline}{2018年7月2日(月) 12:59}

% 提出者名
\newcommand{\studentName}{川上 遼太}
% 提出者所属
\newcommand{\studentAff}{基礎工学部 情報科学科 計算機科学コース 3年}
% 提出者学籍番号
\newcommand{\studentID}{09B16022}
% 提出者電子メールアドレス
\newcommand{\studentEmail}{u407773h@ecs.osaka-u.ac.jp}
% 課題提出日
\newcommand{\submitted}{2018年7月1日(日)}


\begin{document}

%Title page start
\begin{titlepage}
\mbox{\vspace{10cm}}

\begin{center}
{\Huge\bfseries
\lectureName \\
第\reportNumber{}回レポート} \\
\vspace{1cm}
{\Large \reportTitle}
\vspace{5cm}

{\large
\begin{tabular}{rcl}
担当教員 & : & \teacherName \\
提出者   & : & \studentName \\
所属/学年   & : & \studentAff\\
学籍番号 & : & \studentID \\
電子メール & : & \texttt{\studentEmail}\\
\\
提出日 & : & \submitted \\
締切日 & : & \deadline
\end{tabular}
}
\end{center}
\end{titlepage}
%Title page end

\tableofcontents
\newpage

\section{}


\appendix % appendixは、「これ以降は付録」ということを表す
\section{section name}



\end{document}


%%%%%%%%%%%% 図 %%%%%%%%%%%%
% \begin{figure}[H]
% \centering
% \includegraphics[width=.65\textwidth]{.eps}
% \caption{図の名称}
% \label{fig:}
% \end{figure}
%
%%%%%%%%%%%% 表 %%%%%%%%%%%%
% \begin{tab}[H]
% \centering
% \begin{tabu}{|c|c|l|}
% \hline
% \multicolumn{1}{|c|}{} & \multicolumn{1}{c|}{} & \multicolumn{1}{c|}{} \\
% \hline
%  &  &  \\
%  &  &  \\
%  &  &  \\
%  &  &  \\
% \hline
% \end{tabu}
% \caption{表の名称}
% \label{tab:}
% \end{tab}

% 表がはみ出た時
% http://www.inashiro.com/2011/02/11/latex-intro-7/
% \scalebox{0.6}[0.8]{
% }
